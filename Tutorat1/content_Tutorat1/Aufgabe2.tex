%!Tex Root = ../Tutorat1.tex
% ./Packete.tex
% ./Design.tex
% ./Deklarationen.tex
% ./Aufgabe1.tex
% ./Aufgabe3.tex
% ./Bonus.tex

\section{Task 2 - Communication}

\setcounter{task}{1}

\begin{frame}[fragile]{Task 2 - Communication}{Task 2}
  \begin{solution}
    \begin{columns}
      \begin{column}{0.8\textwidth}
        \begin{itemize}
          \item Baudrate of $115200 \frac{bits}{s}$
          \item We require 16 clock periods to sample $1 bit$
          \item Required clock frequency = $115200 \frac{bits}{s} \cdot \frac{16}{bits} = 1.8432 MHz$
        \end{itemize}
      \end{column}
    \end{columns}
  \end{solution}
\end{frame}

\if\preview0{
\begin{frame}[fragile]{Task 2 - Communication}{Task 2}
  \begin{solution}
    \begin{columns}
      \begin{column}{0.8\textwidth}
        \begin{itemize}
          \item We have a 48MHz clock and want to reduce it with division factor x
          \item Our goal speed is 1.8432 MHz
          \item The ration between the given and required clock is $\frac{48MHz}{1.8432MHz} = 26.04167$
        \end{itemize}
      \end{column}
    \end{columns}
  \end{solution}
\end{frame}

\begin{frame}[fragile]{Task 2 - Communication}{Task 2}
  \begin{solution}
    \begin{columns}
      \begin{column}{0.8\textwidth}
        \begin{itemize}
          \item For each byte of actual data we send: 1 start bit + 8 data bits + 1 parity bit + 2 stop bits = 12 bits
          \item We have a total of 10MB = $10 \cdot 2^{20}$ bytes of data to send
          \item This takes $\frac{(10 \cdot 2^{20} \cdot 12)bits}{115200bits/s} = 1092.27s$ time
        \end{itemize}
      \end{column}
    \end{columns}
  \end{solution}
\end{frame}

\begin{frame}[fragile, allowframebreaks]{Task 2 - Communication}{Task 2}
  \begin{solutionnoinc}
    \begin{itemize}
      \item \alert{transmission rates sender and receiver with unknown clock frequency $F_r$:}
      \begin{itemize}
        \item $r_s=\frac{48 \cdot 10^6}{16 \cdot 26} \mathrm{~Hz}, \quad r_r=\frac{F_r}{16 \cdot 26}$
      \end{itemize}
      \item \alert{the sender transmits the $k$th symbol during time:}
      \begin{itemize}
        \item $\left(\frac{(k-1)}{r_s}, \frac{k}{r_s}\right)$
      \end{itemize}
      \item \alert{receiver samples the $k$th symbol at time:}
      \begin{itemize}
        \item $\frac{k-0.5}{r_r}$
      \end{itemize}
      \item \alert{For correct reception of the kth symbol, it has to be constant for a clock cycle before and after the sample time:}
      \begin{itemize}
        \item $\left(\frac{k-0.5}{r_r}-\frac{1}{r_r \cdot 16}, \frac{k-0.5}{r_r}+\frac{1}{r_r \cdot 16}\right)$
      \end{itemize}
    \end{itemize}
  \end{solutionnoinc}

  \begin{solution}
    \begin{itemize}
      \item \alert{If receiver is slower than the sender, the second stop bit is ‘critical’. To be sampled correctly, it has to be sampled before the second stop bit ends:}
      \begin{itemize}
        \item $\begin{aligned}[t] \frac{12-0.5}{r_r}+\frac{1}{r_r \cdot 16} & \leq \frac{12}{r_s} \\ r_r & \geq \frac{11.5 \cdot 16+1}{12 \cdot 16} \cdot r_s = \frac{\frac{11.5 \cdot 16+1}{12 \cdot 16} \cdot 48 \cdot 10^6}{16 \cdot 26} \mathrm{~Hz}=\frac{\boxed{46.25 \mathrm{MHz}}}{16 \cdot 26} \end{aligned}$
      \end{itemize}
      \item \alert{If receiver is faster than the sender, the first stop is ‘critical’. To be sampled correctly, it has to be sampled after the start of the first stop bit:}
      \begin{itemize}
        \item $\begin{aligned}[t] \frac{11-0.5}{r_r}-\frac{1}{r_r \cdot 16} & \geq \frac{10}{r_s} \\ r_r & \leq \frac{10.5 \cdot 16-1}{10 \cdot 16} \cdot r_s= \frac{\frac{10.5 \cdot 16-1}{10 \cdot 16} \cdot 48 \cdot 10^6}{16 \cdot 26} \mathrm{~Hz}=\frac{\boxed{50.1 \mathrm{MHz}}}{16 \cdot 26} \end{aligned}$
      \end{itemize}
      \item \alert{Combining these two inequalities, we get the valid range of clock frequencies for the receiver:}
      \begin{itemize}
        \item $46.25 \mathrm{MHz} \leq F_r \leq 50.1 \mathrm{MHz}$
      \end{itemize}
    \end{itemize}
  \end{solution}
\end{frame}
% \fi
