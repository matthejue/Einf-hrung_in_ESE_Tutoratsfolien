\def\pascal{0}
\def\preview{1}

\input{./content_Tutorat1/Packete}
\input{./content_Tutorat1/Design}
\input{./content_Tutorat1/Deklarationen}

\includeonly{
  ./content_Tutorat1/Aufgabe1,
  ./content_Tutorat1/Aufgabe2,
  ./content_Tutorat1/Aufgabe3,
  ./content_Tutorat1/Aufgabe4,
  ./content_Tutorat1/Bonus,
}

\begin{document}

\begin{withoutheadline}
  \begin{withoutfootline}
    \begin{frame}
      \titlepagesecond
    \end{frame}
  \end{withoutfootline}

  \begin{frame}[shrink=10]{Gliederung}
    \tableofcontents[hideallsubsections]
  \end{frame}
\end{withoutheadline}

\section{Organisation}

\setcounter{section}{-1}

\if\pascal0{
  \begin{frame}[allowframebreaks, fragile]{Organisation}
    \begin{itemize}
      \item \alert{feedback for me:} \url{https://forms.gle/f3YN8EFrZ1vsfPoC6}
      \begin{figure}
        \centering
        \includegraphics[height=0.3\paperheight]{./figures/feedback.png}
      \end{figure}
    \end{itemize}
  \end{frame}
}\fi

\section{Overview Aperiodic Task Scheduling}

\begin{frame}{Overview Aperiodic Task Scheduling}
  \centering
  \includegraphics[width=\textwidth]{./figures/overview_aperiodic_task_scheduling.png}
\end{frame}

% \begin{frame}{TT Cyclic Executive Scheduling}{Recap: Definitions}
%     \begin{itemize}
%         \item $\Gamma:$ set of all periodic tasks
%         \item $\tau_i:$ one particular periodic task (the i-th)
%         \item $\tau_{i,j}:$ the $j$th instance of task $i$
%         \item $r_{i,j}:$ release time of $j$th instance of task $i$
%         \item $d_{i,j}:$ absolute deadline of the $j$th instance of task $i$
%         \item $\Phi_i:$ phase of task $i$
%         \item $D_i:$ relative deadline of task $i$
%     \end{itemize}
% \end{frame}
%
% \begin{frame}{TT Cyclic Executive Scheduling}{Recap: Three assumptions}
% \begin{enumerate}
%     \item The instances of a periodic task are regularly activated at a constant rate. The interval between two consecutive activations is called period. The release times satisfy $r_{i,j} = \Phi_i + (j-1)T_i$
%     \item All instances have the same worst case execution time $C_i$ (also written as $WCET(i)$)
%     \item All instances of a periodic task have the same relative deadline $D_i$. Therefore the absolute deadlines satisfy $d_{i,j} = \Phi_i + (j-1)T_i + D_i$
% \end{enumerate}
% \end{frame}

\section{Aufgabe 1}

\begin{frame}[fragile]{Test}{Test}
  \begin{itemize}
    \item \inlinebox*{testscript}
  \end{itemize}
\end{frame}

%!Tex Root = ../Tutorat4.tex
% ./Packete.tex
% ./Design.tex
% ./Deklarationen.tex
% ./Aufgabe1.tex
% ./Aufgabe3.tex
% ./Bonus.tex

\section{Task 2}

\setcounter{task}{1}

\begin{frame}[allowframebreaks]{Task 2}{Schedulability Test for Fixed Priorities – Rate Monotonic (RM)}
  \begin{itemize}
    \item asdff
  \end{itemize}
\end{frame}

%!Tex Root = ../Tutorat6.tex
% ./Packete.tex
% ./Design.tex
% ./Deklarationen.tex
% ./Aufgabe1.tex
% ./Aufgabe2.tex
% ./Bonus.tex

\section{Task 3}

\setcounter{task}{1}

%!Tex Root = ../Tutorat7.tex
% ./Packete.tex
% ./Design.tex
% ./Deklarationen.tex
% ./Aufgabe1.tex
% ./Aufgabe2.tex
% ./Aufgabe3.tex
% ./Bonus.tex

\section{Task 4}

\setcounter{task}{1}

\begin{frame}[allowframebreaks]{Task 4}{List Scheduling}
  \begin{tasknoinc}
    \begin{figure}
      \centering
      \includegraphics[height=0.6\paperheight]{./figures/task4_sequence_graph.png}
    \end{figure}
  \end{tasknoinc}
  \framebreak
  \begin{solutionnoinc}
    \begin{figure}
      \centering
      \includegraphics[height=0.5\paperheight]{./figures/task4_schedule_empty.png}
    \end{figure}
  \end{solutionnoinc}
  \framebreak
  \begin{solution}
    \begin{figure}
      \centering
      \includegraphics[height=0.5\paperheight]{./figures/task4_schedule.png}
    \end{figure}
  \end{solution}
  \framebreak
  \begin{solution}
    \begin{itemize}
      \item $L=8$
    \end{itemize}
  \end{solution}
  \framebreak
  \begin{solutionnoinc}
    \begin{itemize}
      \item Multiplier because the critical path ($1 \rightarrow 6 \rightarrow 8 \rightarrow 10 \rightarrow 11$) is not delayed by adder but multiplier.
    \end{itemize}
  \end{solutionnoinc}
  \framebreak
  \begin{solutionnoinc}
    \centering
    \tiny
    \begin{tabular}{c|c|l|l|l|}
    \hline$t$ & $k$ & $U_{t, k}$ & $T_{t, k}$ & $S_{t, k}$ \\
    \hline \multirow{2}{*}{0} & $r_1$ & & & \\
    \cline { 2 - 5 } & $r_2$ & & & \\
    \hline \multirow{2}{*}{1} & $r_1$ & & & \\
    \cline { 2 - 5 } & $r_2$ & & & \\
    \hline \multirow{2}{*}{2} & $r_1$ & & & \\
    \cline { 2 - 5 } & $r_2$ & & & \\
    \hline \multirow{2}{*}{3} & $r_1$ & & & \\
    \cline { 2 - 5 } & $r_2$ & & & \\
    \hline \multirow{2}{*}{4} & $r_1$ & & & \\
    \cline { 2 - 5 } & $r_2$ & & & \\
    \hline \multirow{2}{*}{5} & $r_1$ & & & \\
    \cline { 2 - 5 } & $r_2$ & & & \\
    \hline \multirow{2}{*}{6} & $r_1$ & & & \\
    \cline { 2 - 5 } & $r_2$ & & & \\
    \hline \multirow{2}{*}{7} & $r_1$ & & & \\
    \cline { 2 - 5 } & $r_2$ & & & \\
    \hline
    \end{tabular}
  \end{solutionnoinc}
\end{frame}

\begin{frame}{Task 4}{List Scheduling}
  \begin{solutionnoinc}
    \centering
    \tiny
    \begin{tabular}{c|c|l|l|l|}
    \hline$t$ & $k$ & $U_{t, k}$ & $T_{t, k}$ & $S_{t, k}$ \\
    \hline \multirow{2}{*}{0} & $r_1$ & v1 v2 v3 v4 & - & v1 v2 \\
    \cline { 2 - 5 } & $r_2$ & v5 & - & v5 \\
    \hline \multirow{2}{*}{1} & $r_1$ & v3 v4 & - & v3 v4 \\
    \cline { 2 - 5 } & $r_2$ & v6 & v5 & v6 \\
    \hline \multirow{2}{*}{2} & $r_1$ & v7 & - & v7 \\
    \cline { 2 - 5 } & $r_2$ & v9 & v6 & v9 \\
    \hline \multirow{2}{*}{3} & $r_1$ & v8 & - & v8 \\
    \cline { 2 - 5 } & $r_2$ & - & v9 & - \\
    \hline \multirow{2}{*}{4} & $r_1$ & v10 & - & v10 \\
    \cline { 2 - 5 }  & $r_2$ & - & - & - \\
    \hline \multirow{2}{*}{5} & $r_1$ & - & - & - \\
    \cline { 2 - 5 } & $r_2$ & v11 & - & v11 \\
    \hline \multirow{2}{*}{6} & $r_1$ & - & - & - \\
    \cline { 2 - 5 } & $r_2$ & - & v11 & - \\
    \hline \multirow{2}{*}{7} & $r_1$ & & & \\
    \cline { 2 - 5 } & $r_2$ & & & \\
    \hline
    \end{tabular}
  \end{solutionnoinc}
\end{frame}

\begin{frame}{Task 4}{List Scheduling}
  \begin{solution}
    \begin{itemize}
      \item $L=7$
    \end{itemize}
  \end{solution}
\end{frame}

\begin{frame}{Task 4}{List Scheduling}
  \begin{solutionnoinc}
    \centering
    \tiny
    \begin{tabular}{c|c|l|l|l|}
    \hline$t$ & $k$ & $U_{t, k}$ & $T_{t, k}$ & $S_{t, k}$ \\
    \hline \multirow{2}{*}{0} & $r_1$ & & & \\
    \cline { 2 - 5 } & $r_2$ & & & \\
    \hline \multirow{2}{*}{1} & $r_1$ & & & \\
    \cline { 2 - 5 } & $r_2$ & & & \\
    \hline \multirow{2}{*}{2} & $r_1$ & & & \\
    \cline { 2 - 5 } & $r_2$ & & & \\
    \hline \multirow{2}{*}{3} & $r_1$ & & & \\
    \cline { 2 - 5 } & $r_2$ & & & \\
    \hline \multirow{2}{*}{4} & $r_1$ & & & \\
    \cline { 2 - 5 } & $r_2$ & & & \\
    \hline \multirow{2}{*}{5} & $r_1$ & & & \\
    \cline { 2 - 5 } & $r_2$ & & & \\
    \hline \multirow{2}{*}{6} & $r_1$ & & & \\
    \cline { 2 - 5 } & $r_2$ & & & \\
    \hline \multirow{2}{*}{7} & $r_1$ & & & \\
    \cline { 2 - 5 } & $r_2$ & & & \\
    \hline
    \end{tabular}
  \end{solutionnoinc}
\end{frame}

\begin{frame}{Task 4}{List Scheduling}
  \begin{solutionnoinc}
    \centering
    \tiny
    \begin{tabular}{c|c|l|l|l|}
    \hline$t$ & $k$ & $U_{t, k}$ & $T_{t, k}$ & $S_{t, k}$ \\
    \hline \multirow{2}{*}{0} & $r_1$ & v1 v2 v3 v4 & - & v1 v2 v3 v4\\
    \cline { 2 - 5 } & $r_2$ & v5 & - & v5 \\
    \hline \multirow{2}{*}{1} & $r_1$ & v7 & - & v7 \\
    \cline { 2 - 5 } & $r_2$ & v6 & v5 & v6 \\
    \hline \multirow{2}{*}{2} & $r_1$ & - & - & - \\
    \cline { 2 - 5 } & $r_2$ & v9 & v6 & v9 \\
    \hline \multirow{2}{*}{3} & $r_1$ & v8 & - & v8 \\
    \cline { 2 - 5 } & $r_2$ & - & v9 & - \\
    \hline \multirow{2}{*}{4} & $r_1$ & v10 & - & v10 \\
    \cline { 2 - 5 }  & $r_2$ & - & - & - \\
    \hline \multirow{2}{*}{5} & $r_1$ & - & - & - \\
    \cline { 2 - 5 } & $r_2$ & v11 & - & v11 \\
    \hline \multirow{2}{*}{6} & $r_1$ & - & - & - \\
    \cline { 2 - 5 } & $r_2$ & - & v11 & - \\
    \hline \multirow{2}{*}{7} & $r_1$ & & & \\
    \cline { 2 - 5 } & $r_2$ & & & \\
    \hline
    \end{tabular}
  \end{solutionnoinc}
\end{frame}
\begin{frame}[allowframebreaks]{Task 4}{List Scheduling}
  \begin{solutionnoinc}
    \begin{figure}
      \centering
      \includegraphics[height=0.6\paperheight]{./figures/task4_sequence_graph_edit.png}
    \end{figure}
  \end{solutionnoinc}
  \begin{solution}
    \begin{itemize}
      \item $L=7$
    \end{itemize}
  \end{solution}
\end{frame}

%!Tex Root = ../Tutorat1.tex
% ./Packete.tex
% ./Design.tex
% ./Deklarationen.tex
% ./Aufgabe1.tex
% ./Aufgabe2.tex
% ./Bonus.tex

\section{Bonus}

\begin{frame}[fragile,allowframebreaks]{Bonus}{Hexadecimal System\vspace{0.5cm}}
  \begin{itemize}
    \item \alert{Example:} $\begin{aligned}[t]
        \underline{beef}_{16} &= 11 \cdot 16^3 + 14 \cdot 16^2 + 14 \cdot 16^1 + 15 \cdot 16^0 \\
        &= 11 \cdot 4096 + 14 \cdot 256 + 14 \cdot 16 + 15 \\
        &= 48879
      \end{aligned}$
  \end{itemize}

  \centering
  \begin{itemize}
    \item \alert{all Bin and Hex assigned:}
  \end{itemize}
  \begin{terminal}
  0    1    2    3    4    5    6    7    8    9
  ---- ---- ---- ---- ---- ---- ---- ---- ---- ----
  0000 0001 0010 0011 0100 0101 0110 0111 1000 1001

  A    B    C    D    E    F
  ---- ---- ---- ---- ---- ----
  1010 1011 1100 1101 1110 1111
  \end{terminal}
  \framebreak
  \begin{itemize}
    \item \alert{Hex $\Rightarrow$ Bin:}
  \end{itemize}
  \begin{terminal}
     D    4    F    6    6    E
  1101 0100 1111 0110 0110 1110
  \end{terminal}
  \begin{itemize}
    \item \alert{Bin $\Rightarrow$ Hex:}
  \end{itemize}
  \begin{terminal}
  1101 0100 1111 0110 0110 1110
     D    4    F    6    6    E
  \end{terminal}
  % https://tex.stackexchange.com/questions/13380/explicit-frame-break-with-beamer-class
  \framebreak
  \begin{itemize}
    \item \alert{Derivation:}
    \begin{itemize}
      \item $\begin{aligned}[t]
          a4_{16} &= 10 \cdot 16^1 + 4 \cdot 16^0 \\
                  &= 10 \cdot {(2^4)}^1 + 4 \cdot {(2^4)}^0 \\
                  &= 1010_2 \cdot 2^4 + 0100_2 \cdot 1 \\
                  &= (1000_2 \cdot 2^4 + 10_2 \cdot 2^4) + (100_2 \cdot 2^0) \\
                  &= (1 \cdot 2^7 + 1 \cdot 2^5) + (1 \cdot 2^2) \\
                  &= 1010\_0100_{2}
        \end{aligned}$
      \item \alert{idea:} shifting a number works in hexadecimal system $1a_{16} \cdot 10^2_{16} = 1a00$ decimal system $17 \cdot 10^2 = 1700$ and binary system $11_2 \cdot 10_2^2 = 1100_2$ quite similar.
      \item but beacuse $16 = 2^4$ the \alert{hexadecimal} and \alert{binary system} are particulary easy to convert into each other.
    \end{itemize}
  \end{itemize}
\end{frame}

\begin{frame}{Bonus}{Quiz question}
  % https://tex.stackexchange.com/questions/190988/beamer-replace-one-word-with-another
  \begin{itemize}
    \item How many bits will a hex number with 5 symbols have in binary system?
      \begin{itemize}
        \item[$\square$] 10
        \item[$\square$] 15
        \alt<2>{\item[$\blacksquare$]}{\item[$\square$]} 20
        \item[$\square$] 16
      \end{itemize}
    \item<2>\alert{example:} 0xa\_aaaa = 0b1010\_1010\_1010\_1010\_1010
    \item<2>the \alert{binary number} has $4$ times as many bits because $16=2^4$
  \end{itemize}
\end{frame}


% \section{Literatur}
%
% \begin{frame}{Bücher}
%   \printbibliography[type=book,heading=subbibliography,title={Bücher}]
% \end{frame}
%
% \begin{frame}{Artikel}
%   \printbibliography[type=article,heading=subbibliography,title={Artikel}]
% \end{frame}
%
% \begin{frame}{Vorlesungen}
%   \printbibliography[type=unpublished,heading=subbibliography,title={Vorlesungen}]
% \end{frame}
%
% \begin{frame}{Online}
%   \printbibliography[type=online,heading=subbibliography,title={Online}]
% \end{frame}
%
% \begin{frame}{Sonstiges}
%   \printbibliography[nottype=book, nottype=article, nottype=online, nottype=unpublished,heading=subbibliography,keyword=wikikeyword,title={Sonstige Quellen}]
% \end{frame}
\end{document}
