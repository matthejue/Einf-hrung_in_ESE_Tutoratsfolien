%!Tex Root = ../Tutorat3.tex
% ./Packete.tex
% ./Design.tex
% ./Deklarationen.tex
% ./Aufgabe1.tex
% ./Aufgabe3.tex
% ./Aufgabe4.tex
% ./Bonus.tex

\section{Task 2}

\setcounter{task}{1}

\begin{frame}[shrink=10]{Task 2}{Latest Deadline First}
  \vspace{0.5cm}
  \begin{task}
    \centering
    \includegraphics[height=0.7\paperheight]{./figures/2_tab_graph.png}
  \end{task}
\end{frame}

\begin{frame}{Task 2}{Latest Deadline First}
  \begin{requirements}
    \begin{itemize}
      \item is \alert{non-preemptive}
      \item \alert{synchronous task activations}
      \item tasks are \alert{dependent}, use \alert{precedence graph}, going from \alert{tail} to \alert{head}
      \item $max(D_i)$ ($d_i$ if $\forall J_i\in J: a_i=c \wedge c\ne 0$) for all tasks $J_i$ \alert{without successors} or whose \alert{successors} have been all selected in the \alert{precedence graph} inserted into the queue to be \alert{executed last}
      \item at runtime, tasks are extracted from the \alert{head of the queue:} the \alert{first task} inserted in the queue will be \alert{executed last}
        % select the tasks with the \alert{latest deadline} to be scheduled last.
      \item \alert{minimizes} the \alert{maximum lateness}
    \end{itemize}
  \end{requirements}
\end{frame}

\begin{frame}{Task 2}{Latest Deadline First}
  \centering
  \includegraphics[height=0.7\paperheight]{./figures/2_steps.png}
\end{frame}

\begin{frame}{Task 2}{Latest Deadline First}
  \begin{itemize}
    \item \alert{queue of tasks:} (\qquad,\qquad,\qquad,\qquad,\qquad,\qquad,\qquad,\qquad)
  \end{itemize}
  \centering
  \includegraphics[width=0.7\textwidth]{./figures/2_empty.png}
\end{frame}

\begin{frame}{Task 2}{Latest Deadline First}
  \begin{itemize}
    \item \alert{queue of tasks:}  $(J_1, J_2, J_3, J_5, J_4, J_6, J_7, J_8)$
  \end{itemize}
  \centering
  \includegraphics[width=0.7\textwidth]{./figures/2_sol.png}
\end{frame}
