%!Tex Root = ../Tutorat4.tex
% ./Packete.tex
% ./Design.tex
% ./Deklarationen.tex
% ./Aufgabe2.tex
% ./Aufgabe3.tex
% ./Aufgabe4.tex
% ./Bonus.tex

\section{Task 1}

\setcounter{task}{1}

\begin{frame}{Mixed Task Sets}{}
    \begin{itemize}
        \item So far: we differentiated between \alert{periodic} and \alert{aperiodic} tasks.
        \item Now: Consider a \alert{mixed} task set!
        \item We want to be able to find a schedule when there's both \alert{periodic} and \alert{aperiodic} tasks.
    \end{itemize}
\end{frame}

\begin{frame}{Schedulability tests}{Sufficient? Necesarry?}
    \begin{itemize}
        \item We're interested in whether a given problem can be scheduled by algorithms.
        \item Depending on the algorithm we can derive sufficient and necesarry conditions.
        \item[]\alert{Sufficient:} If $A \implies B$ then A is a sufficient condition for B.
        \item[]\alert{Necesarry:} If $B \implies A$ then A is a necesarry condition for B.
        \item A necesarry and sufficient condition means, both statements are logically equivalent.
    \end{itemize}
\end{frame}
\begin{frame}{Schedulability tests}{Utilization}
    Different kind of utilizations also play a big role in our analysis. We introduced the \alert{processor utilization factor} $U = \sum\limits_{i=1}^{n}\cfrac{C_i}{T_i}$ and later on $U_s$ as the server utilization.

    (More about servers later)
\end{frame}
\begin{frame}{RM - Rate Monotonic Scheduling}{Schedulability}
\begin{itemize}
    \item RM is optimal among all fixed-priority assignments in the sense that no other fixed-priority algorithm can schedule a task set that cannot be scheduled by RM.
    \item As in the lecture, we have $\sum\limits_{i=1}^{n}\cfrac{C_i}{T_i} \leq n(2^{1/n}-1)$ as a \alert{sufficient} but not \alert{necesarry} condition.
\end{itemize}
\end{frame}
\begin{frame}{RM(PS) - Rate Monotonic Polling Server}
    \begin{itemize}
        \item One way to handle both periodic and aperiodic tasks is to use a so called server.
        \item This PS (Polling Server) acts as a periodic task (meaning it is instantiated at regular intervals $T_s$) whose job it is to, once it has the highest priority, serve any pending aperiodic requests within the limits of a server capacity $C_s$.
        \item Since we introduce yet another periodic task, the schedulability analysis simply is the same as normal $RM$ with one additional task. Again, we have the \alert{sufficient} but not \alert{necesarry} condition: $\cfrac{C_s}{T_s} + \sum\limits_{i=1}^{n}\cfrac{C_i}{T_i} \leq (n+1)(2^{1/(n+1)}-1)$
    \end{itemize}
\end{frame}

\begin{frame}{EDF - Total Bandwidth Server}

\end{frame}
